\documentclass{article}
\usepackage[utf8]{inputenc}
\usepackage{fullpage}
\usepackage[russian]{babel}
\usepackage{amsthm,amsmath,amsfonts,amssymb}
\usepackage{listings}
\usepackage{xcolor}
\usepackage{listings}
\usepackage{tikz}
\usepackage{graphicx}
\usepackage{gensymb}
\usepackage{hyperref}
\usepackage{caption}
\usepackage{float}
\usepackage{subcaption} 
\usetikzlibrary{graphs}
\input{paging.tex}


% CS + BI

%\newcommand{\mod}{\texttt{mod}}
\renewcommand{\H}{\mathcal{H}}
\newcommand{\F}{\mathbb{F}}
\newcommand{\Z}{\mathbb{Z}}
\newcommand{\N}{\mathbb{N}}

\newcommand{\lcm}{\texttt{lcm}}
\newcommand{\Prb}[1]{\underset{#1}{\textbf{Pr}}}
\newcommand{\Prbb}[2]{\underset{#1}{\textbf{Pr}}\left[ \; #2 \;\right]}
\newcommand{\Ex}[1]{\underset{#1}{\textbf{E}}}

\renewcommand{\O}{\mathcal{O}}
\renewcommand{\t}[1]{\texttt{#1}}

\lstset{
    language=Python,
    commentstyle=\color{gray!80!white}
}

\title{Практика по алгоритмам}
\author{Александр Мишунин, Михаил Слабодкин\footnote{Составители сборника не являются авторами самих задач. Авторы не указаны в учебных целях.}}
\date{Осень, 2018}

% \keeppages{8}

\begin{document}

\maketitle

\pagebreak \input{practice/p01.tex}
%\pagebreak \input{practice/p02.tex}
%\pagebreak \input{practice/p03.tex}
%\pagebreak \input{practice/p04.tex}
%\pagebreak \input{practice/p05.tex}
%\pagebreak \input{practice/p06.tex}
%\pagebreak \input{practice/p07.tex}
%\pagebreak \input{practice/p08.tex}
%\pagebreak \input{practice/p09.tex}
%\pagebreak \input{practice/p10.tex}
%\pagebreak \input{practice/p11.tex}
%\pagebreak \input{practice/p12.tex}
%\pagebreak \input{practice/p13.tex}
%\pagebreak \input{practice/p14.tex}

\end{document}
