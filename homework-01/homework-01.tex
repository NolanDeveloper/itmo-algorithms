\documentclass{article}

\usepackage[utf8]{inputenc}
\usepackage{fullpage}
\usepackage[russian]{babel}
\usepackage{amssymb}
\usepackage{mathtools}

\DeclarePairedDelimiter{\ceil}{\lceil}{\rceil}

\begin{document}

\section{Домашняя работа(Мироев Д.Ш.)}

\newcommand{\limit}[0]{\displaystyle{\lim_{x \to \infty}}}

\begin{enumerate}

\item Эквивалентны ли следующие факты?

\begin{enumerate}
\item $f = \Theta(g)$
\item $\limit \frac{f(n)}{g(n)} = \mathcal{O}(1)$ 
\end{enumerate}

Из (b) следует (a):

\begin{itemize}
\item $\limit \frac{f(n)}{g(n)} = C(> 0)$
\item $\forall \varepsilon>0\ \exists N\ \forall n>N : \left|\frac{f(n)}{g(n)}-C\right|<\varepsilon$
\item $\forall \varepsilon>0\ \exists N\ \forall n>N : -\varepsilon<\frac{f(n)}{g(n)}-C<\varepsilon$
\item $\forall \varepsilon>0\ \exists N\ \forall n>N : C-\varepsilon<\frac{f(n)}{g(n)}<C+\varepsilon$
\item $\forall \varepsilon>0\ \exists N\ \forall n>N : (C-\varepsilon)g(n)<f(n)<(C+\varepsilon)g(n)$
\item $\exists N, C_1(>0), C_2(>0) \forall n>N\ : (C-\varepsilon)g(n)<f(n)<(C+\varepsilon)g(n)$
\item $f(n) = \Theta(g(n))$
\end{itemize}

Но из (a) не следует (b). Пример:

$f(n)=n\ \mathrm{mod}\ 2$ \\
$g(n)=1$

$n\ \mathrm{mod}\ 2 = \Theta(1)$,\\
но $\limit \frac{n\ \mathrm{mod}\ 2}{1}$ не существует.

\item Дайте ответ для двух случаев $\mathbb{N} \rightarrow \mathbb{N}$ и $\mathbb{N} \rightarrow \mathbb{R}_{>0}$:

Рассуждения будут аналогичны для двух случаев.

\begin{enumerate}
\item Если в определении $\mathcal{O}$ опустить условие про $\mathbb{N}$ (т.е.
оставить просто $\forall n$), будет ли полученное определение эквивалентно исходному?

$\exists N, C(>0)\ \forall n \geqslant N : f(n) \leqslant C \cdot g(n)$\\
Рассмотрим $A = \mathrm{max} \{\frac{f(n)}{C\cdot g(n)} \mid n < N\}$\\
и $C_{1} = C \cdot A$.\\
Тогда $\exists C_{1}(>0)\ \forall n\ : f(n) \leqslant C_{1} \cdot g(n)$.\\
При $n \geqslant N$ это очевидно. Рассмотрим $n < N$:\\
$f(n) = (C \cdot g(n)) \cdot \frac{f(n)}{C \cdot g(n)} \leqslant (C \cdot g(n)) A = C_{1} \cdot g(n)$

Ответ: можно.

\item Тот же вопрос про $o$.

Рассмотрим пример $f(n) = n$ и $g(n) = n^2$.\\
$\forall C(>0)\ \exists N\ \forall n(>N) : n < Cn^2$\\
$\forall C(>0)\ \exists N\ \forall n(>N) : n - Cn^2 < 0$\\
$\forall C(>0)\ \exists N\ \forall n(>N) : n(1 - Cn) < 0$ \\
$\forall C(>0)\ \exists N\ \forall n(>N) : 1 - Cn < 0$\\
$\forall C(>0)\ \exists N\ \forall n(>N) : n > \frac{1}{C}$\\
Отсюда $N = \ceil[\Big]{\frac{1}{C}}$

Ответ: нельзя.

\end{enumerate}

\item Продолжим отношение ``$\preceq$'' на функциях до отношения на классах
эквивалентности по отношению ``$\sim$'' эквивалентности введённому на паре.
Правда ли, что получится отношение линейного порядка? То есть $\forall x, y :
(x \preceq y) \vee (y \preceq x)$.

В классе мы доказали разумность определения ``$\preceq$'' для классов
эквивалентности по отношению ``$\sim$'', то есть

$\forall f, f' (f \sim f') \forall g, g' (g \sim g') : (f \preceq g) \rightarrow (f' \preceq g')$.

Поэтому можно вместо отношения на классах рассмотреть отношение на представителях.
Но для функции ``$\preceq$'' не является отношением линейного порядка:

$f(n) = 
     \begin{cases}
       \text{1,} &\text{если n-\text{чётн.}}\\
       \text{n,} &\text{если n-\text{нечётн.}}\\
     \end{cases} 
$

$
g(n) = 
     \begin{cases}
       \text{1,} &\text{если n-\text{нечётн.}}\\
       \text{n,} &\text{если n-\text{чётн.}}\\
     \end{cases} 
$

$f \neq \mathcal{O}(g)$ и $g \neq \mathcal{O}(f)$.

Значит и для классов ``$\preceq$'' не является отношением линейного порядка.

\item Покажите, что: $g(n) = o(f (n)) \Rightarrow f (n) + g(n) = \Theta(f (n))$.

Доказательство:

$g(n) = o(f(n)) \Leftrightarrow \forall C(>0) \exists N \forall n(>N) : g(n) < C \cdot f(n)$\\
$g(n) = o(f(n)) \Leftrightarrow \forall C(>0) \exists N \forall n(>N) : g(n) + f(n) < (C + 1) \cdot f(n)$\\

\end{enumerate}

\end{document}
